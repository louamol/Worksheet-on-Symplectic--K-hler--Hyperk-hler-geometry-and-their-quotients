\documentclass{worksheetclass}

\usepackage{import}
\import{}{custom_macros.tex}

\title{Geometry and Quotients}

% DOCUMENT -----------------------------

\begin{document}

\maketitle

\tableofcontents

\pagebreak

\section{Symplectic manifolds}

    \subsection{Symplectic structure}

        Let $M$ be a smooth manifold.    
        \begin{defn}
            A $2$-form $\omega$ on $M$ is said to be \emph{non-degenerate}\index{non-degenerate form} if is satisfies the following property: if there exists $X_p\in T_pM$ such that $\omega(X_p,Y_p)=0$ for all $Y_p\in T_pM$ then $X_p=0$.
        \end{defn}
        
        \begin{prop}
            A $2$-form $\omega$ is non-degenerate if and only if the matrix formed by its components $\omega_{ab}$ such that $\omega=\omega_{ab}\d x^a\wedge\d x^b$ is invertible.
        \end{prop}
        
        In odd dimensions, skew-symmetric matrices are always singular, the requirement that $\omega$ be non-degenerate therefore implies that $M$ must have even dimension. The existence of a symplectic form therefore strongly constrains the manifold.
    
        \begin{defn}
            A \emph{symplectic form}\index{symplectic form} $\omega$ on a smooth manifold $M$ is a closed non-degenerate $2$-form. $(M,\omega)$ is then called a \emph{symplectic manifold}\index{manifold!symplectic}\index{symplectic manifold}.
        \end{defn}
        \begin{defn}
            A diffeomorphism between two symplectic manifolds $f:(M,\omega)\to(M',\omega')$ is called a \emph{symplectomorphism}\index{symplectomorphism} if $f^*\omega' = \omega$. The set of symplectomorphisms from $M$ to $M$ form a pseudo-group\footnote{A pseudo-group is a set of diffeomorphisms between open sets of a space, satisfying group-like and sheaf-like properties. It is a generalisation of the concept of a group.}, called the \emph{symplectomorphism group}\index{symplectomorphism group}.
        \end{defn}

        \begin{defn}
            A vector field $X\in\mathfrak{X}(M)$ is called \emph{symplectic}\index{symplectic vector field} if
            \begin{equation*}
                \L_X\omega = 0.
            \end{equation*}
        \end{defn}
        \begin{prop}
            A vector field $X\in\mathfrak{X}(M)$ is symplectic if and only if its flow $\phi_t:M\to M$ is a symplectomorphism for every $t\in R$.
        \end{prop}
        This means that the infinitesimal version of symplectomorphisms are symplectic vector fields. These vector fields form a Lie subalgebra of $\mathfrak{X}(M)$.

        \begin{defn}
            Let $(M,\omega)$ be a symplectic manifold and $G$ a group. We say that an action $\phi:G\times M\to M$ is \emph{symplectic}\index{symplectic action} each map $\phi(g,\cdot)$ is a symplectomorhphism.
        \end{defn}
        That is to say, $G$ is a subgroup of the group of symplectomorphisms of $(M,\omega)$.

    \subsection{Group action}

        Let $(M,\omega)$ be a symplectic manifold and $\cdot:G\times M\to M$ a symplectic action of $G$ on $M$. If $\g$ is the Lie algebra of $G$, let
        \begin{equation}
            \langle\cdot,\cdot\rangle : \g^*\times\g\to\R
        \end{equation}
        be the canonical pairing between $\g$ and $\g^*$. 
        
        The first thing to notice is that any $\xi\in\g$ induces a vector field $\rho(\xi)\in\mathfrak{X}(M)$ on $M$ describing the infinitesimal action of $\xi$ on $M$. More precisely, at a point $p\in M$ the tangent vector $\rho(\xi)_p$ is defined as
        \begin{equation}
            \rho(\xi)_p\equiv\eval{\dv{}{t}}_{t=0}(\exp(t\xi)\cdot p),\label{eq:rhodef}
        \end{equation}
        where $\exp:\g\to G$ is the exponential map. Let $\iota_{\rho(\xi)}\omega$ denotes the contraction of this vector field with $\omega$.
        \begin{prop}
            If the action $\cdot$ is symplectic action, then $\iota_{\rho(\xi)}\omega$ is closed for all $\xi\in\g$.
        \end{prop}
        \begin{proof}
            This can be seen directly by using the \emph{Cartan formula} $\mathcal{L}_X\omega=\d(\iota_X\omega)+\iota_X\d\omega$, which holds for every differential form $\omega$ and any vector field $X$.
        \end{proof}

        The $1$-form $\iota_{\rho(\xi)}\omega$ is therefore always closed, but not necessarily exact. It is exact only for a subclass of symplectic actions, which we know turn to.

        \begin{defn}
            The $G$-action is said to be \emph{Hamiltonian}\index{Hamiltonian group action} if and only if the following conditions hold:
            \begin{itemize}
                \item for every $\xi\in\g$ the one-form $\iota_{\rho(\xi)}\omega$ is exact, meaning that
                \begin{equation}
                    \iota_{\rho(\xi)}\omega=\d H_{\xi}
                \end{equation}
                for some function $H_{\xi}\in C^\infty(M,\R)$. If this holds, then one may choose the $H_\xi$ to make the map $H:\xi\mapsto H_\xi$ linear. Note that this just means that $\rho(\xi)$ is a Hamiltonian vector field with Hamiltonian $H_\xi$.
                \item the map
                \begin{equation}
                    H:\left(
                    \begin{array}{ccc}
                            \g & \longrightarrow & C^\infty(M,\R) \\
                            \xi & \longmapsto & H_\xi
                    \end{array}
                    \right)
                \end{equation}
                is a Lie algebra homomorphism, $C^\infty(M,\R)$ being understood as the algebra of smooth functions on $M$ with the Poisson bracket.
            \end{itemize}
        \end{defn}

    \subsection{Momentum map}
        
        Let us consider a Hamiltonian action of a Lie group $G$ on $(M,\omega)$. The $1$-form $\iota_{\rho(\xi)}\omega$ can then we written as
        \begin{equation}
            \iota_{\rho(\xi)}\omega= \d H_{\xi}
        \end{equation}
        for some function $H_{\xi}\in C^\infty(M,\R)$, and $H:\g\to C^\infty(M,\R)$ is a Lie algebra homomorphism. We then have the following definition.
        \begin{defn}
            A \emph{momentum map}\index{momentum map} for the action of $G$ on $(M,\omega)$ is a map $\mu:M\to\g^*$ such that
            \begin{equation}
                \iota_{\rho(\xi)}\omega=\d(\langle \mu,\xi\rangle)
            \end{equation}
            for all $\xi\in\g$. Here $\langle \mu,\xi\rangle$ is the function from $M$ to $\R$ defined by $\langle \mu,\xi\rangle(p)\equiv\langle \mu(p),\xi\rangle$, which is thing else than the Hamiltonian $H_\xi$.
        \end{defn}
        The momentum map is uniquely defined up to an additive constant of integration. A momentum map is often also required to be $G$-equivariant, where $G$ acts on $\g^*$ via the coadjoint action (dual of the adjoint action). If the group is compact or semisimple, then the constant of integration can always be chosen to make the momentum map coadjoint equivariant. However, in general the coadjoint action must be modified to make the map equivariant (this is the case for example for the Euclidean group). The modification is by a 1-cocycle on the group with values in $\g^*$.

        \begin{examp}
            In the case of the Hamiltonian action of the circle $G=\U(1)$, the Lie algebra dual $\g^*$ is naturally identified with $\R$, and the momentum map is simply the Hamiltonian function that generates the circle action.
        \end{examp}

        \begin{examp}
            Another classical case occurs when $M$ is the cotangent bundle of $\R^{3}$ and $G$ is the Euclidean group generated by rotations and translations. That is, $G$ is a six-dimensional group, the semidirect product of $\SO(3)$ and $\R^{3}$. The six components of the momentum map are then the three angular momenta and the three linear momenta.
        \end{examp}

        \begin{prop}
            We have the following properties:
            \begin{itemize}
                \item Let $G$ be a group with Hamiltonian action on $(M_1,\omega_1)$ and $H$ be a group with Hamiltonian action on $(M_2,\omega_2)$. Then the natural action of $G\times H$ on $(M_{1}\times M_{2},\omega _{1}\times \omega _{2})$ is Hamiltonian, with momentum map the direct sum of the two momentum maps. Here $\omega _{1}\times \omega _{2}\equiv\pi _{1}^{*}\omega _{1}+\pi _{2}^{*}\omega _{2}$, where $\pi_i : M_1 \times M_2 \rightarrow M_i$ denotes the projection map.
                \item Let $G$ be a group with Hamiltonian action on $(M,\omega)$, and $N$ a submanifold of $M$ invariant under $G$ such that the restriction of the symplectic form on $M$ to $N$ is non-degenerate. This imparts a symplectic structure to $N$ in a natural way. Then the action of $G$ on $N$ is also Hamiltonian, with momentum map the composition of the inclusion map with $M$'s momentum map.
            \end{itemize}
        \end{prop}

    \subsection{Symplectic quotient}\label{sec:symplecticquotient}

        Suppose that the action of a compact Lie group $G$ on the symplectic manifold $(M,\omega)$ is Hamiltonian, as defined above, with momentum map $\mu:M\to\g^*$. From the Hamiltonian condition, it follows that the zero-level set of the momentum map
        \begin{equation}
            M_0\equiv\mu^{-1}(0)
        \end{equation}
        is invariant under $G$.

        Assume now that $0$ is regular value of $\mu$ and that $G$ acts freely and properly\footnote{Note that any continuous action of a Lie group on a smooth manifold is proper.} on $M_0$. Thus $M_0$ and its quotient $M_0/G$ are both manifolds. The quotient inherits a symplectic form from $M$. That is, there is a unique symplectic form on the quotient whose pullback to $M_0$ equals the restriction of $\omega$ to $M_0$. Thus the quotient is a symplectic manifold.

        \begin{defn}
            The symplectic manifold $M_0/G$ is called \emph{symplectic quotient}\index{symplectic quotient} (also called \emph{Marsden-Weinstein quotient}\index{Marsden-Weinstein quotient} or \emph{symplectic reduction}\index{symplectic reduction}) of $M$ by $G$ and is denoted $M//G$\index{$//$}.
        \end{defn}
        Its dimension equals the dimension of $M$ minus twice the dimension of $G$:
        \begin{equation}
            \dim(M//G)=\dim M-2\dim G.
        \end{equation}

        Note that the construction carries over if one replaces the zero-level set $M_0$ by the preimage $\mu^{-1}(\xi)$ under $\mu$ of some non-zero invariant element $\xi$ of $\g^*$ (which corresponds to the residual freedom in the definition of the momentum map), or even by the premimage of the whole orbit of an element of $\g^*$ under $G$.

        \begin{examp}
            Let us consider the group $G=\R$ acting on $(\R^4,\d x^1\wedge\d x^2+\d x^3\wedge\d x^4)$ as
            \begin{equation}
                (x^1,x^2,x^3,x^4) \mapsto (x^1,x^2,x^3,x^4+t).
            \end{equation}
            where $t$ denotes the coordinate on $G$. I.e., $G$ is the group of translations in the $x^4$-coordinate, which are clearly symplectomorphisms. The most general element of $\mathfrak{g}$ is $\lambda \p_t$ with $\lambda\in\R$. On can see that
            \begin{equation}
                \rho(\lambda\p_t)=\p_{x^4}.
            \end{equation}
            It follows that
            \begin{equation}
                \iota_{\rho(\lambda\p_t)}=-\lambda\d x^2.
            \end{equation}
            We know want to find a momentum map $\mu:\R^4\to\mathfrak{g}^*$ such that
            \begin{align}
                \d\langle\mu,\lambda\p_t\rangle &= -\lambda\d x^2\\
                \Leftrightarrow\qquad \langle\mu,\lambda\p_t\rangle &= -\lambda x^2 + C
            \end{align}
            where $C\in\R$ is an arbitrary constant, this gives
            \begin{equation}
                \mu = (C-x^2)\d t.
            \end{equation}
            The zero-level set of $\mu$ is therefore $M_0=\{(x^1,x^2,x^3,x^4)\in M|x^2=C\}$, which can be identified with $\R^3$. $0$ is clearly a regular value of $\mu$ since $\mu_*=-\d x^2$ which does not vanish on $M_0$, or anywhere as a matter of fact. It is also clear that $G$ acts freely on $M_0$, so we can quotient $M_0$ by $G$, i.e. project $M_0$ through the $x^4$ direction. The resulting space is diffeomorphic to $\R^2$ and we have the projection
            \begin{equation}
                \pi:\left(
                \begin{array}{ccc}
                        M_0\cong\R^3 & \longrightarrow & M_0/G\cong\R^2 \\
                        (x^1,x^2,C,x^4) & \longmapsto & (x^1,x^2)
                \end{array}
                \right).
            \end{equation}
            The only symplectic structure on $M_0/G$ whose pullback through $\pi$ is $\omega|_{M_0}$\footnote{Note that this $2$-form is never a sympletic form because it is always degenerate.} is $\omega'=\d x^1\wedge\d x^2$. Finally, we found that
            \begin{equation}
                (\R^4,\d x^1\wedge\d x^2+\d x^3\wedge\d x^4)//\R=(\R^2,\d x^1\wedge\d x^2).
            \end{equation}
        \end{examp}

\section{Riemannian quotients}

    Before moving to Kähler geometry which combines complex geometry, riemannian geometry and symplectic geometry, let us just say a few words the riemannian quotient. We start with a riemannian manifold $(M,g)$ and a Lie group $G$ which acts on $M$. We can call this action \emph{isometric} if it acts through isometries, in which case $G$ is a subgroup of the isometry group of $(M,g)$. For any $\xi\in\g$, one can again define a vector field $\rho(\xi)$ again as \eqref{eq:rhodef}. They are the infinitesimal generators of the isometry generated by $\xi$, i.e. $\rho(\xi)$ is a Killing vector. Up untill now, the situation is completely analog to the symplectic case.
    
    If $G$ acts freely on $M$, one can construct a canonical metric $g'$ on the quotient manifold $M/G$, by requiring that the projection
    \begin{equation}
        \pi:M\to M/G
    \end{equation}
    is a riemannian submersion. The metric $g'$ on the quotient in obtained as follows: if $X',Y'$ are vector fields on $M/G$, they can be horizontally lifted\footnote{Let $\pi:E\to M$ over $M$ be a smooth fibre bundle. The vertical bundle is the kernel $VE=\text{ker}(\pi_*)$ of the tangent map $\pi_*:TE\to TM$. Since $\pi_{*e}$ is surjective at each point $e\in E$, so $VE$ is a regular subbundle of $TE$. An Ehresmann connection on $E$ is a choice of a complementary subbundle $H$E to $VE$ in $TE$, called the horizontal bundle of the connection. At each point $e$ in $E$, the two subspaces form a direct sum, such that $T_eE = V_eE\oplus H_eE$.} to $X,Y$, i.e. choosing two vector fields $X,Y$ orthogonal to the Killing vectors and projecting to $X',Y'$. We then define $g'(X',Y')=g(X,Y)$. This ensure that the projection from $M_0$ to $M/G$ is a riemannian submersion.

    Note that the pullback of the metric $g'$ to $M_0$ is not the restriction of $g$ to $M_0$, sinced this pulled-backs ymmetric form is degenerate along the action of the group.

\section{Kähler manifolds}

    \subsection{Complex structure}

        \begin{defn}
            A \emph{linear complex structure} on a vector space $V$ is linear map $J:V\to V$ such that $J^2=-\text{id}_V$ (such that its eigenvalues are $\pm i$).
        \end{defn}
        
        \begin{defn}
            An \emph{almost complex structure} $J$ on a smooth manifold $M$ is the data a linear complex structure $J_p$ on each tangent space $T_pM$ which varies smoothly. In other words, it is a map $J:TM\to TM$, i.e. a $(1,1)$-tensor, such that $J^2=-\id_{TM}$. $(M,J)$ is then called an \emph{almost complex manifold}.
        \end{defn}
        
        It is important to notice that imposing the existence of an almost complex structure strongly constrains the manifold. If $M$ admits the latter, it must be even-dimensional and orientable.

        One can show that any even-dimensional vector space admits a linear complex structure. Therefore, an even-dimensional manifold always admits a map $J:TM\to TM$ tensor pointwise such that $J^2=-\id_{TM}$. Only when this local tensor can be patched together to be defined globally does the pointwise linear complex structure yield an almost complex structure, which is then uniquely determined.

        \begin{examp}
            If $M=\R^2$, then $\left\{\p_x,\p_y\right\}$ forms a basis of $TM$. Let us define an almost complex structure $J$ on $\R^2$ by
            \begin{equation}
                J\left( \p_x \right)=\p_{y},\qquad J\left( \p_{y} \right)=-\p_{x},
            \end{equation}
            therefore $J$ here rotates $\R^2$ anticlockwise through $\pi/2$ around the origin.
        \end{examp}

        The reason these structures are said to be ``almost'' complex and not ``fully'' complex is the following: they are not necessarily related to complex manifolds\footnote{A complex manifold is a manifold with an atlas of charts to the open unit disc in $\mathbb{C}^n$ such that the transition maps are holomorphic.}. Let us be more precise. It is clear that every complex manifold is in particular an almost complex manifold. Indeed, given some holomorphic coordinates $z^\mu=x^\mu+iy^\mu$, we can always define an almost complex structure $J$ by
        \begin{equation}
            J\left(\p_{x^\mu}\right)=\p_{y^\mu},\qquad J\left(\p_{y^\mu}\right)=-\p_{x^\mu}\label{eq:Jcanform1}
        \end{equation}
        or, equivalently,
        \begin{equation}
            J\left(\p_{z^\mu}\right)=i\p_{z^\mu},\qquad J\left(\p_{\bar{z}^\mu}\right)=-i\p_{\bar{z}^\mu}\label{eq:Jcanform2}.
        \end{equation}
        The canonical almost complex structure $J$ defined by \eqref{eq:Jcanform1} for a complex manifold is then said to be \emph{induced} by the complex structure.
        \begin{defn}
            Coordinates such that the almost complex structure of a manifold takes the canonical form (\ref{eq:Jcanform1}) in an entire neighborhood of a point $p\in M$ are called \emph{local holomorphic coordinates} for $J$ at $p$.
        \end{defn}
        
        The converse question however, whether the almost complex structure implies the existence of a complex structure is much less trivial and not true in general. For any almost complex manifold $(M,J)$, one can always construct coordinates around a point $p\in M$ such that the almost complex structure $J$ takes the canonical form at the point $p$. However, it is in general not possible to find coordinates such takes this form in an entire neighborhood of $p$, i.e. local holomorphic coordinates. This is only possible for a subset of almost complex structures.
        \begin{defn}
            An almost complex structure $J$ is said to be \emph{integrable} if $M$ can be covered by local holomorphic coordinates for $J$, i.e. if there is complex structure on $M$ that induces $J$.
        \end{defn}

        \begin{prop}
            If an almost complex manifold $(M,J)$ is integrable, then the patches of local holomorphic coordinates can be patch together to form an atlas for $M$, therefore giving it a complex structure. Moreover, this complex structure induces $J$.
        \end{prop}

        In conclusion:
        \begin{result}
            Integrable almost complex structures and complex structures are equivalent in the sense that one induces the other and vice-versa.
        \end{result}

        Finally, let us quickly discuss some consequences of having an complex structure. Given a complex manifold $M$ of complex dimension $n$, its tangent bundle as a smooth vector bundle is a real rank $2n$ vector bundle $TM$ on $M$. More precisely, it is the complexification $\displaystyle TM\otimes \mathbb{C} \to M$ of the real tangent bundle $\displaystyle TM \to M$. The fibres of $TM\otimes \mathbb{C}$ are the vector spaces $T_pM\otimes\C\to M$. Recall that
        \begin{equation*}
            T_pM\otimes \C\text{ on }\R ~\cong~ T_pM\text{ on }\C.
        \end{equation*}
        To summarize, the complexified tangent bundle in the vector bundle whose fibres are the complexified vector spaces. We use the notation $iX_p\equiv X_p\otimes i$ for any $X_p\in T_pM$\footnote{Note that it is really important to have a tensor product (instead of a Cartesian product for example), in order to have properties like $fX+igX = (f+ig)X$.}. 
        
        We have seen that having a complex structure is equivalent to having a integrable almost complex structure $J$. The endomorphisms $J_p$ are initially defined on the tangent spaces but they can be extended complex-linearly to endomorphisms on the complexified tangent spaces by defining $J(X_p+iY_p)=J(X_p)+iJ(Y_p)$ for $X_p,Y_p\in T_pM$. At each point $p\in M$, $J$ can then be used to split the tangent space into its \emph{holomoprhic} and \emph{antiholomoprhic} parts:
        \begin{align}
                T^{1,0}_pM &= \left\{X_p\in T_pM|J(X_p)=iX_p\right\},\label{eq:tangentspacesplit1}\\
                T^{0,1}_pM &= \left\{X_p\in T_pM|J(X_p)=-iX_p\right\}.\label{eq:tangentspacesplit2}
        \end{align}
        Tangent vectors are dubbed holomorphic or antiholomorphic accordingly. Any tangent vector $X_p$ can be decomposed in the sum of a holomorphic and an antiholomorhic part as
        \begin{equation}
            X_p = \underbrace{\frac{1}{2}(1-iJ)X_p}_{\text{holom.}}+\underbrace{\frac{1}{2}(1+iJ)X_p}_{\text{antiholom.}}.
        \end{equation}

        This splitting of the tangent spaces induces a splitting of the whole complexified tangent bundle:
        \begin{equation}
            TM\otimes \mathbb{C} =T^{1,0}M\oplus T^{0,1}M.
        \end{equation}
        The fibers of $T^{1,0}M$ and $T^{0,1}M$ being precisely \eqref{eq:tangentspacesplit1} and \eqref{eq:tangentspacesplit2}.

        Moreover, the splitting of the tangent bundle induces a splitting of the cotangent bundle and therefore also of all the tensor spaces, which are constructed from $TM$ and $T^*M$. In particular, the space of complex $k$-forms $\Omega^k(M,\C)={\textstyle\bigwedge^k V}\otimes\C$ can be decomposed as
        \begin{equation*}
            \Omega^k(M,\C) = \bigoplus_{p+q=k}\Omega^{(p,q)}(M)
        \end{equation*}
        where
        \begin{equation*}
            \Omega^{(p,q)}(M) = \bigwedge^pT^{*1,0}M\bigwedge^qT^{*0,1}M
        \end{equation*}
        in the space of $(p,q)$-forms on $M$. This is the \emph{Hodge decomposition}\index{Hodge decomposition}. We have the decomposition
        \begin{equation}
            \d= \p + \bp
        \end{equation}
        into the exterior derivative in the holomorphic and antiholomorphic directions.
        \begin{examp}
            If $M=\R^2=\C$, we have
            \begin{equation}
                \d=\d x \pdv{}{x}+\d y \pdv{}{y} = \d z\pdv{}{z}+\d\bz\pdv{}{\bz}=\p+\bp
            \end{equation}
            where $\d z=\d x+i\d y$ and $\d\bz =\d x+i\d y$.
        \end{examp}
        In particular, we have
        \begin{equation}
            0 = \d^2 = \p^2 + (\p\bp+\bp\p) + \bp^2
        \end{equation}
        but these operators lies in different space:
        \begin{align}
            \p^2&:\Omega^{p,q}\to\Omega^{p+2,q},\\
            \p\bp+\bp\p&:\Omega^{p,q}\to\Omega^{p+1,q+1},\\
            \bp^2&:\Omega^{p,q}\to\Omega^{p,q+2}
        \end{align}
        so they must be zero separately:
        \begin{equation}
            \p^2=0,\qquad \p\bp+\bp\p=0,\qquad \bp^2=0.
        \end{equation}
        The usual de Rham cohomology also undergoes this splitting, giving rise to the Dolbeault cohomology.

        In a local holomorphic chart $\phi=(z^1,\dots,z^n):U\subset M\to\C^n$, one has distinguished real coordinates $(x^{1},\dots ,x^{n},y^{1},\dots ,y^{n})$ defined by $z^j=x^j+iy^j$ for each $j=1,\dots,n$. These give complex-valued vector fields (that is, sections of the complexified tangent bundle),
        \begin{equation}
            \pdv{}{z^j}=\frac{1}{2}\left( \pdv{}{x^j}-i\pdv{}{y^j} \right),\qquad \pdv{}{\bz^j}=\frac{1}{2}\left( \pdv{}{x^j}+i\pdv{}{y^j} \right).
        \end{equation}
        Taken together, these vector fields form a frame for $TM\otimes \mathbb{C}|_U$, the restriction of the complexified tangent bundle to $U$. As such, these vector fields also split the complexified tangent bundle into two subbundles
        \begin{equation}
            T^{1,0}M|_U\equiv\text{span}\left\{\pdv{}{z^j}\right\},\qquad  T^{0,1}M|_U\equiv\text{span}\left\{\pdv{}{\bz^j}\right\}.
        \end{equation}
        Under a holomorphic change of coordinates, these two subbundles of $TM\otimes \mathbb{C}|_U$ are preserved, and so by covering $M$ by holomorphic charts one obtains a splitting of the complexified tangent bundle. This is precisely the splitting into the holomorphic and anti-holomorphic tangent bundles previously described. From this perspective, the name holomorphic tangent bundle becomes transparent because the transition functions for the holomorphic tangent bundle are also holomorphic. Thus the holomorphic tangent bundle is a genuine holomorphic vector bundle.

        Dual to those vector fields, are complex-valued one-forms $\d z^j=\d x^j+i\d y^j,\d\bar{z}^j=\d x^j-i\d y^j$ on $U$. In terms of components, the Hodge decomposition tells us that any $\omega\in\Omega^{(p,q)}(M)$ can be written as
        \begin{equation}
            \omega = \omega_{a_1\dots a_p\bar{b}_1\dots\bar{b}_q}(z,\bz)~\d z^{a_1}\wedge\dots\wedge \d z^{a_p}\wedge \d\bz^{\bar{b}_1}\wedge\dots\wedge \d\bz^{\bar{b}_q}
        \end{equation}
        where the functions $\omega_{a_1\dots a_p\bar{b}_1\dots\bar{b}_q}(z,\bz)$ have $p$ holomorphic and $q$ antiholomorphic indices. 
        
        

    \subsection{Kähler structure}

        \begin{defn}
            An almost complex structure $J$ is said to be \emph{compatible}\index{almost complex structure!compatible} with the symplectic form $\omega$ if 
            \begin{equation*}
                \omega(JX,JY) = \omega(X,Y)
            \end{equation*}
            for all $X,Y\in TM$.
        \end{defn}
        This compatibility condition has two very important consequences:
        \begin{prop}
            If $J$ is compatible with $\omega$, then the bilinear form $g$ on the tangent space of $M$ defined as
            \begin{equation}
                g_p(X_p,Y_p)=\omega_p(X_p,JY_p)\label{eq:kahlermetric}
            \end{equation}
            is symmetric, positive-definite and non-degenerate. In other words, $g$ is a riemannian metric.
        \end{prop}
        This implies that given a symplectic structure and a compatible almost complex structure, we also naturally have riemannian structure. We now have every thing in hand to define the notion of Kähler manifold.
        \begin{defn}
            A \emph{Kähler manifold}\index{manifold!Kähler} $(M,\omega,J)$, is a symplectic manifold $(M,\omega)$ equipped with a compatible, integrable almost complex structure $J$.
        \end{defn}
        A Kähler manifold is therefore the union of three structures: a riemannian structure, a symplectic structure and a complex structure. It is the fact the we are provided with a compatible complex structure that allows us to construct a metric from the symplectic form. We can therefore use the three points of view at will.

        Even though we started with an symplectic structure $\omega$ and a compatible almost complex structure $J$ to induce a riemannian structure $g$, we do not have to proceed in that order. Indeed, an important fact about Kähler manifolds is that the data of any two structure out of $\omega,J$ and $g$ uniquely fixes the last one.
        \begin{prop}
            Given any two out of the three structures (complex structure $J$, symplectic form $\omega$, metric $g$), it uniquely fixes the last one by imposing compatibility.
        \end{prop}
        We could therefore have made other equivalent choices for our definition.
        
        The second important consequence of the compatibility condition is the following.
        \begin{prop}
            The condition $\omega(JX,JY)=\omega(X,Y)$ also says that under the decomposition
            \begin{equation}
                \Omega^2(M) = \Omega^{2,0}\oplus\Omega^{1,1}\oplus\Omega^{0,2}
            \end{equation}
            of $2$-forms into their holomorphic and antiholomorphic parts, the symplectic form $\omega$ actually lies in $\Omega^{1,1}$.
        \end{prop}
        In particular, one must have $\p\omega=0$ and $\bp\omega=0$ separately. The complex version of the Poincaré lemma therefore states that $\omega$ can always be locally written as
        \begin{equation}\label{eq:Kählerpot}
            \omega = i\p\bp K 
        \end{equation}
        for some function $K\in C^\infty (M,\C)$.
        \begin{defn}
            The function $K$ locally defining the symplectic form on a Kähler manifold as in (\ref{eq:Kählerpot}) is called the \emph{Kähler potential}\index{Kähler potential}.
        \end{defn}
        Notice that $K$ is defined up to the transformations
        \begin{equation}
            K(z,\bz)\mapsto K(z,\bz)+f(z)+f(\bz)
        \end{equation}
        where $f$ is holomorphic.
        
        
        Since the metric is determined by the symplectic form, see \eqref{eq:kahlermetric} and that the symplectic form is determined Kähler potential,   the components of the metric  are also fully determined by the Kähler potential:
        \begin{equation}
            g_{a\bar{b}}=\p_a\p_{\bar{b}}K
        \end{equation}
        in terms of local holomorphic coordinates $(z^a,\bz^a)$ and were $K$ defines the symplectic form. For such a metric, one can show that the only non-vanishing coefficient of the Levi-Civita connection are
        \begin{align}
            \Gamma^a_{bc} &= g^{a\bar{d}}\p_bg_{c\bar{d}},\\
            \Gamma^{\bar{a}}_{\bar{b}\bar{c}} &= g^{\bar{a}d}\p_{\bar{b}}g_{d\bar{c}}.
        \end{align}
        \begin{result}
            A symplectic structure on a complex manifold implies that the symplectic form necessarily comes from a Kähler potential, and so does the metric. The Kähler potential is defined up to translation by holomorphic or antiholomorphic functions.
        \end{result}
        
        \begin{examp}
            Let us treat $\C^n$ as a Kähler manifold. The Kähler potential associated to the flat metric
            \begin{equation}
                g = \sum_a \left( (\d x)^2+(\d y)^2 \right) = \sum_a \delta_{a\bar{a}}\d z^a\d\bz^{\bar{a}}
            \end{equation}
            on $\R^{2n}$ is
            \begin{equation}
                K(z,\bz) = \sum_a\abs{z^a}^2
            \end{equation}
            and the symplectic form is
            \begin{equation}
                \omega = \sum_i \d x^i\wedge\d y^i = \sum_a \delta_{a\bar{a}}\d z^a\wedge\d\bz^{\bar{a}}.
            \end{equation}
        \end{examp}

        \begin{examp}
            We can also treat $\C\mathbb{P}^n$ as a Kähler manifold with the Kähler potential
            \begin{equation}
                K(z,\bz) = \ln\left( 1+\sum^n_{a=1}\abs{z^a}^2 \right)
            \end{equation}
            on the coordinate patch $\C^n\subset\C\mathbb{P}^n$ (it covers the complement of an hyperplane in $\C\mathbb{P}^n$). The resulting metric is called the \emph{Fubini-Study metric}\index{Fubini-Study metric}.
        \end{examp}

    \subsection{Kähler quotients}

        Let us consider the action $\cdot:G\times M\to M$ of a Lie group $G$ on a Kähler manifold $(M,\omega,J,g)$. Since a Kähler manifold is a special case of symplectic manifold, the symplectic quotient construction applies and yields a symplectic form on $M_0/G$ if the action of $G$ is symplectic.

        Let $K$ be a Killing vector of $(M,J,g)$, i.e. $K\in\mathfrak{X}(M)$ such that
        \begin{equation}
            \L_Kg=0\qquad\text{ and }\qquad \L_KJ=0.
        \end{equation}
        One can then show that $K$ is then also a Killing vector on $M_0$ with the restricted metric $g$.

\section{Hyperkähler manifolds}

    \subsection{Quaternionic structure}

    \subsection{Hyperhähler structure}

        A hyperkähler manifold is a manifold (necessarily of dimension a multiple of four) which admits an action on the tangent space of the $i,j$ and $k$ (quaternions) in a manner which is compatible with the metric.
        \begin{defn}
            A \emph{hyperkähler manifold}\index{hyperkähler manifold} is a riemannian manifold $(M,g)$ provided with three orthogonal automorphisms $I,J,K:TM\to TM$ of the tangent bundle that are covariant constant with respect to the Levi-Civita connection, i.e.
            \begin{equation*}
                \nabla I=\nabla J=\nabla K = 0,
            \end{equation*}
            and which satisfy the quaternionc identities
            \begin{equation*}
                I^2=J^2=K^2=IJK = -1.
            \end{equation*}
        \end{defn}
        These condition imply in particular
        \begin{itemize}
            \item each these tangent bundle automorphisms define an integrable complex structure on $M$;
            \item the metric $g$ is Kähler with respect to all three;
            \item the three Kähler forms $\omega_1,\omega_2$ and $\omega_3$ are therefore closed and give three symplectic structures on $M$.
        \end{itemize}

        We can also have the following, equivalent, definition in terms of $G$-structure:
        \begin{defn}
            The \emph{symplectic group}\emph{symplectic group} $\Sp(2n,F)$\index{$\Sp$} is a classical group defined as the set of linear transformations that of a $2n$-dimensional symplectic vector space $V$ over the field $F$ which preserve the non-degenerate skew-symmetric bilinear form (symplectic point at a point). If the latter is denoted $\Omega$, then
            \begin{equation*}
                \Sp(2n,F)\equiv\{M\in M_{2n\times2n}(F)|m^T\Omega M=\Omega\}.
            \end{equation*}
            We use the notation $\Sp(n)\equiv\Sp(n,\R)$.
        \end{defn}
        In the case where
        \begin{equation}
            \Omega=
            \begin{bmatrix}
                0 & \mathbbm{1}_n \\
                \mathbbm{1}_n & 0
            \end{bmatrix},
        \end{equation}
        we have $\Sp(2n,F)=\SL(2,F)$.
        \begin{defn}[alternative definition]
            A \emph{hyperkähler manifold}\index{hyperkähler manifold} is a quaternionic manifold with a torsion-free $\Sp(n)$-structure. 
        \end{defn}
        A hyperkähler manifold is simultaneously a hypercomplex manifold and a quaternionic Kähler manifold. From this definition, every hyperkähler manifold $M$ has a $2$-sphere of complex structures (i.e. integrable almost complex structures) with respect to which the metric is Kähler. In particular, it is a hypercomplex manifold, meaning that there are three distinct complex structures, $I, J$, and $K$, which satisfy the quaternion relations
        \begin{equation}
            I^2=J^2=K^2=IJK=-1.
        \end{equation}
        Thus closing the loop with our first definition.

        \begin{remark}
            Recall that a riemannian manifold provided withjust one such automorphism of the tangent bundle is a Kähler manifold. The name ``hyperkähler'' comes from the fact that the metric is Kählerian for several complex strctures. There is, however, an essential difference bewteen Kähler and hyperkähler manifolds. A Kähler matric on a given complex manifold  can be modofied to another Kähler metric simply by adding a hermician form $\p\bp f$ to the initial metric, for a sufficiently small smooth function $f$. Thus the space of Kähler metrics is infinite-dimensional. It is morevoer easy to find examples of Kähler manifolds. For example, any complex submanifold of $\C\P^n$ inherits a Kähler metric. By contrast, hyperkähler metrics are much more rigid. On a compact manifold, if one such emtric exists, then up to isometry there is only a finite-dimensional space of them. Nor is it easy to find examples. Certainly we wil never find then as queternionic submanifolds of the quaternionc projective space $\mathbb{H}\P^n$.
        \end{remark}
        The group $\Sp(n)$ is also an intersection of $\U(2n)$ and $\Sp(2n\C)$, the transformations of $\C^{2n}$ which preserve a non-degenrate skew-symmetric form. Thus a hyperkähler manifold is naturally a complex manifold with a holomorphic sympectic form. One can see this explicitely by taking three Kähler two-forms
        \begin{align}
            \omega_1(X,Y) &= g(IX,Y)\\
            \omega_2(X,Y) &= g(JX,Y)\\
            \omega_3(X,Y) &= g(KX,Y)\\
        \end{align}
        defined for the complex strcture $I,J$ and $K$. With repsect tothe complex structure $I$, the complex form
        \begin{equation}
            \omega_c \equiv \omega_2+i\omega_3
        \end{equation}
        is non-degenerate and covaraint constant, hence closed and holomorphic.

        This point of view provides guidance in the search of exmaples of hyperkähler manifodls, and elucidates the sort of differential equation which needs to be solved. In the first place the holomorphic volume form $\omega_c^n$ must give a covariant constant trvialization of the canonical line bundle. The curvature of this bundl for any Kähler matric is the Ricci form and so hyperkähler metric has in particular vanishing tensor.
        \begin{prop}
            All hyperkähler manifolds are Ricci-flat.
        \end{prop}
        In the lowest dimension, i.e. four dimensions, this means that such metrics satisfy the riemannian version of the Einstein vacuum equations. We have the even more useful result:
        \begin{prop}
            In four dimensions, a simply connected riemannain manifold admit a hyperkähler metric if the riemannian curvature tensor is either self-dual or anti-self-dual.
        \end{prop}
        Consequently, a complete hyperkähler $4$-manifold is a self-dual, euclidean, solution to Einstein's equations. This is very usefull to construct ALE spaces, see section \ref{sec:ALEhyperkahlerquot}.

        Given a compact Kähler manifold with holomorphically trivial canonical line bundle, the Calabi-Yau theorem provides the existence of a Kähler matric with vanishing Ricci tensor. Futhermore, a mush older theorem (Bochner) shows that any holomoprhic form on a compact Kähler manifold with zero Ricci tensor is covariant constant. Therefore, for every copact Kähler manifold  with a holomorphic symplectic form, an apllication of these two theorems yields a hyperkähler metric on the same manifold. This satisfactiry state of affairs can be used the prove the existence of hyperkähleron many exmaples of complex manifolds. The must fundamental is the $K3$ surface -- the only non-trivial example in four realdimensions. In higher dimensions, the Hilbert scheme of zero cycles on a $K3$-surface or a $2$-dimensional complex torus yields anatural class of holomorphic symplectic manifolds and hence hyperkähler metrics.

        \begin{theorem}[Kurnosov, Soldatenkov, Verbitsky]
            The cohomology of any compact hyperkähler manifold embeds into the cohomology of a torus, in a way that preserves the Hodge structure.
        \end{theorem}

        \begin{result}
            \textbf{Hyperkähler manifolds.} A hyperkähler manifold is a manifold which admits an action on the tangent space of the $i,j$ and $k$ in a manner which is compatible with the metric. All hyperkähler manifolds and are Ricci-flat so in four dimensions (lowest dimension for a hyperkähler manifold), this means that the metric satisfies the euclidean version of the Einstein vacuum equations. Morevover, in four dimensions, a simply connected riemannain manifold admit a hyperkähler metric if the riemannian curvature tensor is either self-dual or anti-self-dual.
        \end{result}

        We shall seek something more than existence. We would like to construct solutions in a more explicit manner, in order to gain better understanding of hyperkähler manifolds and to experience the richness of their geometry. There are two main routes to constructing hyperkähler metrics:
        \begin{enumerate}
            \item \textbf{Twistor theory.} This approch is based on R. Penrose's original work in relativity. It provides an encoding of the date for such a metric in terms of holomorphic geometry.Deriving global properties of the metric such as completeness is almost impossible. On the other hand, the quotient construction yields this sort of property quite easily, even if it is not as general as the twistor method.
            \item \textbf{Hyperkähler quotients.} This construction arose also out of questions of mathematical physics, in this case supersymmetry. In practical terms there are two ways of using it. The first is a finite-dimensional construction, whereby determining the actual metric involves solving algebraic equations. The second involves the use of the method in infinite dimensions, even though the quotient itself may be finite-dimensional. Here, one needs to solve differential equations to find the metric. They are, however, equations for which in many cases methods of solution are known so that we have in principle more information than an existence theorem.
        \end{enumerate}

        We shall illustrate these constructions by a representative collection of examples which are chosen according to our guiding principle of seeking complex symplectic manifolds. These hyperkähler manifolds are all a priori complex manifolds with holomorphic symplectic forms:
        \begin{itemize}
            \item resolution of rational surface singularities;
            \item coadjoint orbits of complex Lie groups;
            \item spaces of representations of a surface group in a complex Lie group;
            \item space of based loops in complex Lie group.
        \end{itemize}
        The construction of hyperkahler metrics on these spaces is contained in the work of P. B. Kronheimer, S. K. Donaldson, and others. What is perhaps remarkable is that these diverse spaces nearly all inherit a hyperkahler metric through special cases of solutions to the anti-self-dual Yang-Mills equations in $\R^4$. Those physically motivated equations themselves are ultimately based on the identification of $\R^4$ with the quaternions. Hamilton's ghost may yet rest content.

    \subsection{Hyperkähler quotients}

        We saw the definition of the quotient manifold (section \ref{sec:quotmanifold}), the symplectic quotient (section \ref{sec:symplecticquotient}) and the Kähler quotient (section \ref{sec:quotkahler}), we now see the hyperkähler quotient.

        The quotient construction emphasizes not the complex structures but instead the corresponding Kähler forms $\omega_1,\omega_2$ and $\omega_3$. In this case, by contraction with the inverse forms on cotangent vectors, we can recover $I,J$ and $K$ and the metric itself. Put another way, $\Sp(n)$ is the stabilizer of $\omega_1, \omega_2, \omega_3$ whereas $\GL(n, H)$ is the stabilizer of $I,J,K$.

        A hyperkahler manifold can be characterized in a very straightforward manner using these forms:
        \begin{theorem}
            Let $M^{4n}$ be a manifold with $2$-forms $\omega_1,\omega_2,\omega_3$ whose stabiliser in $\GL(4,\R)$ at each point $p\in M$ is conjugate to $\Sp(n)$. Then the forms define a hyperkähler structure if and only if they are closed.
        \end{theorem}
        This theorem places the theory of hyperkahler manifolds firmly within the context of symplectic geometry.

        Hyperkähler quotient is modelled on the symplectic quotient (see section \ref{sec:symplecticquotient}). Recall that if $(M,\omega)$ is a symplectic manifold with symplectic action of Lie group $G$, then under mild assumptions one can define an momentum map (see section \ref{sec:momentummap})
        \begin{equation}
            \mu:M\to\mathfrak{g}^*
        \end{equation}
        The symplectic quotient construction consists of a new symplectic manifold $M//G\equiv\mu^{-1}(0)/G$. The same kind of manipulation can be done for hyperkähler manifolds.

        Suppose that $M$ is a hyperkähler manifolds, with Lie group $G$ acting so as to preserve the three Kähler forms $\omega_1,\omega_2$ and $\omega_3$. We obtain three moment maps $\mu_1,\mu_2$ and $\mu_3$, or equivalently a vector-valued moment map
        \begin{equation}
            \mu:M\to\mathfrak{g^*}\otimes\R^3,
        \end{equation}
        called the \emph{hyperkähler moment map}\index{hyperkähler moment map}.
        \begin{theorem}
            Let $\mathfrak{z}\subseteq\g^*$ be the set of $G$-invariant elements of $\g^*$. If $\zeta\in\mathfrak{z}\otimes\R^3$ is a regular value for the hyperkähler moment map $\mu$, then $\mu^{-1}(\zeta)/G$ is a hyperkähler manifold.
        \end{theorem}
        Note that
        \begin{equation}
            \dim(\mu^{-1}(\zeta)/G) = \dim M - 4\dim G.
        \end{equation}

\section{Kähler geometry and $\mN=1$ field theories}

\section{Hyperkähler geometry and $\mN=2$ field theories}

\end{document}